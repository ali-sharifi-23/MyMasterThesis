% !TeX root=../main.tex
% در این فایل، عنوان پایان‌نامه، مشخصات خود، متن تقدیمی‌، ستایش، سپاس‌گزاری و چکیده پایان‌نامه را به فارسی، وارد کنید.
% توجه داشته باشید که جدول حاوی مشخصات پروژه/پایان‌نامه/رساله و همچنین، مشخصات داخل آن، به طور خودکار، درج می‌شود.
%%%%%%%%%%%%%%%%%%%%%%%%%%%%%%%%%%%%
% دانشگاه خود را وارد کنید
\university{دانشگاه خواجه نصیرالدین طوسی}
% پردیس دانشگاهی خود را اگر نیاز است وارد کنید (مثال: فنی، علوم پایه، علوم انسانی و ...)
\college{برق}
% دانشکده، آموزشکده و یا پژوهشکده  خود را وارد کنید
\faculty{دانشکدهٔ مهندسی برق}
% گروه آموزشی خود را وارد کنید (در صورت نیاز)
\department{گروه کنترل}
% رشته تحصیلی خود را وارد کنید
\subject{مهندسی برق}
% در صورت داشتن گرایش، خط زیر را از حالت کامت خارج نموده و گرایش خود را وارد کنید
% \field{گرابش}
% عنوان پایان‌نامه را وارد کنید
\title{ارائه فرمول‌بندی یکپارچه بهینه‌سازی مبتنی بر گراف به منظور کالیبراسیون و تخمین‌حالت ربات}
% نام استاد(ان) راهنما را وارد کنید
\firstsupervisor{دکتر حمیدرضا تقی راد}
\firstsupervisorrank{استاد}
% \secondsupervisor{} % دکتر راهنمای دوم
% \secondsupervisorrank{استادیار}
% نام استاد(دان) مشاور را وارد کنید. چنانچه استاد مشاور ندارید، دستورات پایین را غیرفعال کنید.
\firstadvisor{دکتر فلیپ کاردو}
\firstadvisorrank{استادیار}
\secondadvisor{دکتر سید احمد خلیل‌پور}
% نام داوران داخلی و خارجی خود را وارد نمایید.
\internaljudge{دکتر داور داخلی}
\internaljudgerank{دانشیار}
\externaljudge{دکتر داور خارجی}
\externaljudgerank{دانشیار}
\externaljudgeuniversity{دانشگاه داور خارجی}
% نام نماینده کمیته تحصیلات تکمیلی در دانشکده \ گروه
\graduatedeputy{دکتر نماینده}
\graduatedeputyrank{دانشیار}
% نام دانشجو را وارد کنید
\name{محمدرضا}
% نام خانوادگی دانشجو را وارد کنید
\surname{دیندارلو}
% شماره دانشجویی دانشجو را وارد کنید
\studentID{40030824}
% تاریخ پایان‌نامه را وارد کنید
\thesisdate{تابستان 1403}
% به صورت پیش‌فرض برای پایان‌نامه‌های کارشناسی تا دکترا به ترتیب از عبارات «پروژه»، «پایان‌نامه» و «رساله» استفاده می‌شود؛ اگر  نمی‌پسندید هر عنوانی را که مایلید در دستور زیر قرار داده و آنرا از حالت توضیح خارج کنید.
%\projectLabel{پایان‌نامه}

% به صورت پیش‌فرض برای عناوین مقاطع تحصیلی کارشناسی تا دکترا به ترتیب از عبارت «کارشناسی»، «کارشناسی ارشد» و «دکتری» استفاده می‌شود؛ اگر نمی‌پسندید هر عنوانی را که مایلید در دستور زیر قرار داده و آنرا از حالت توضیح خارج کنید.
%\degree{}
%%%%%%%%%%%%%%%%%%%%%%%%%%%%%%%%%%%%%%%%%%%%%%%%%%%%
%% پایان‌نامه خود را تقدیم کنید! %%
\dedication
{
{\Large تقدیم به:}\\
\begin{flushleft}{
	\huge
%آنان که با علم خود زندگی آزاد می‌سازند\\
آنان که در پسِ دیوارهای بلند جامعه، مجال بالیدن و آموختن را نیافتند\\
	\vspace{7mm}
}
\end{flushleft}
}
%% متن قدردانی %%
%% این متن را به سلیقه‌ی خود تعییر دهید
\acknowledgement{
اکنون که به یاری پروردگار و با راهنمایی و حمایت اساتید گرانقدر موفق به اتمام این رساله شده‌ام، بر خود واجب می‌دانم که نهایت سپاس و قدردانی را از تمامی عزیزانی که در این مسیر یار و همراه من بوده‌اند، به جا آورم:

نخست، از استاد ارجمندم، دکتر تقی‌راد که با راهنمایی‌های ارزشمند خود در طی این پایان‌نامه همواره پشتیبان من بودند، صمیمانه تشکر می‌کنم.

همچنین از دکتر خلیل‌پور که در تمامی مراحل این تحقیق، با مشاوره‌های خود راهگشای من بودند، صمیمانه قدردانی می‌نمایم.

خالصانه، از تمامی اساتید، معلمان و مدرسانی که در طول دوره‌های مختلف تحصیلی مرا با گوهر دانش آشنا کرده و از سرچشمه علم سیراب نمودند، کمال سپاس را دارم.

از صمیم قلب از دوستان عزیزم، روح‌الله خرم‌بخت، امیرسامان میرجلیلی، محمدمهدی ناظری، دانیال عبدالهی‌نژاد، مهدی وکیلی و دیگر اعضای آزمایشگاه ارس نهایت سپاسگزاری را دارم، چرا که پیشبرد این هدف بدون حضور و حمایت بی‌دریغ این عزیزان ممکن نبود.
}
%%%%%%%%%%%%%%%%%%%%%%%%%%%%%%%%%%%%
%چکیده پایان‌نامه را وارد کنید
\fa-abstract{
در دنیای امروز، تولیدکنندگان ربات‌ها به دنبال تسهیل استفاده و افزایش دقت عملکردی آن‌ها هستند، که این امر شامل کاهش فرآیندهای تکراری مانند کالیبراسیون حسگرها و ربات‌ها می‌شود. از طرفی دیگر، با توجه به اهمیت مکان‌یابی دقیق، تحقیقات اخیر به سوی روش‌های نوین مبتنی بر گراف و ترکیب داده‌های حسگرهای مختلف سوق یافته است.
در این پایان‌نامه، از این نقطهٔ آشنا آغاز کرده و روش‌های اخیر مورد بررسی قرار می‌گیرد و با استفاده از بستری که الگوریتم‌های مبتنی بر گراف فراهم می‌کنند، گرافی توسعه داده خواهد شد که نه تنها مکان‌یابی را به خوبی انجام دهد، بلکه کالیبراسیون حسگرها و حتی یک گام فراتر، کالیبراسیون ربات را نیز به صورت همزمان و بدون نیاز به فرآیندهای جداگانه انجام دهد. این روش باعث می‌شود تا مسئله‌ای یکپارچه و منعطف ایجاد شود که افزودن هر گونه قید به این مسئله، هیچ‌کدام از فرمول‌بندی‌های پیشین را تحت تأثیر قرار ندهد.
الگوریتم انتخاب شده برای این فرمول‌بندی، گراف عامل است که توانایی آن در مدیریت پیچیدگی‌های محاسباتی به‌واسطه‌ی ماتریس‌های بهینه و بهره‌گیری از معماری گرافی، باعث تسریع در همگرایی حل و افزایش پایداری نتایج می‌شود. علاوه بر این، گراف‌های عامل امکان ادغام آسان داده‌های ناهمگن از حسگرهای مختلف را فراهم می‌کنند، که منجر به افزایش دقت و کاهش وابستگی به یک منبع خاص داده می‌شود.
برای صحت‌سنجی روش پیشنهادی، یک ربات کابلی موازی فروتحریک معلق در نظر گرفته‌ شده‌است. ابتدا مسئله‌ای یکپارچه در فضای گراف ایجاد می‌شود که کالیبراسیون حسگر بینایی، مکان‌یابی و کالیبراسیون سینماتیکی ربات را به صورت همزمان و بدون نیاز به فرآیندهای پیشین انجام دهد تا مفهوم آسان-نصب تحقق یابد. سپس، برای بررسی انعطاف‌پذیری و ارزیابی گراف خود، معادلات شکم‌دهی کابل که از پیچیده‌ترین معادلات در فضای رباتیک است را به گراف اضافه کرده و نتایج بررسی خواهدشد. نتایج نشان‌دهنده‌ی توانمندی بالای این روش در ایجاد یک فرمول‌بندی منعطف است که نه تنها سرعت انجام فرآیند را افزایش داده، بلکه دقت مکان‌یابی و کالیبراسیون را نیز بهبود می‌بخشد.
}
% کلمات کلیدی پایان‌نامه را وارد کنید
\keywords{گراف‌ عامل، کالیبراسیون، مکان‌یابی، ربات‌ کابلی، SLAM}
% انتهای وارد کردن فیلد‌ها
%%%%%%%%%%%%%%%%%%%%%%%%%%%%%%%%%%%%%%%%%%%%%%%%%%%%%%
